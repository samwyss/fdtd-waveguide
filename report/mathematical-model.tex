\section{Mathematical Model}
\label{sec:mathmod}

To model an infinitely long 3D waveguide \textit{in silico}, the simulation domain must be divided up into regions where specific mathematical relations hold. In this particular system there are three such regions (1) PEC surrounded dielectric, (2) Total Field / Scattered Field (TF/SF) 1-way source, and (3) Mur Absorbing Boundary Condition (Mur ABC). Regions (2)-(3) are essential in making the waveguide appear infinite in length to a propagating wave. A high level diagram of an infinite, PEC bordered rectangular waveguide can be found in Fig. \ref{fig:model}(a) and a $\hat{y}$ sliced model where said relations hold can be found in Fig. \ref{fig:model}(b). These governing relations are then discretized to formulate time-stepping formulas which allow the system to evolve transiently.

\begin{figure}[t!]  
	\centering
	%the command within the [] sets the width of the figure, stability-condition is the jpg name
	\includegraphics[width=0.9\linewidth]{model} 
	\caption{Diagrams of (a) High-Level PEC Rectangular Waveguide (b) $\hat{y}$-Sliced Model with Labeled Regions}
	\label{fig:model}
\end{figure}

\subsection{Model Formulation}
\label{subsec:model-formulation}

\subsubsection{PEC Surrounded Dielectric}
\label{subsec:dielectric-formulation}

As outlined in Fig. \ref{fig:model}(b) the vast majority of the simulation domain is composed of a PEC enclosed dielectric, the governing equations of which are Amp\`{e}re's and Faraday's Laws respectively. In differential form, these equations take the form 
\begin{align}
    \nabla\times\textbf{H} = \frac{\partial\textbf{D}}{\partial t} + \textbf{J}
    \label{eq:ampere}
\end{align}
and 
\begin{align}
    \nabla\times\textbf{E}=-\frac{\partial\textbf{B}}{\partial t} - \textbf{M}
    \label{eq:faraday}
\end{align}
where \textbf{E} is the electric field, \textbf{D} is the electric flux density, \textbf{H} is the electric field, \textbf{B} is the magnetic flux density, \textbf{J} is the free electric current density, and \textbf{M} is the fictitious free magnetic current density.

For simplicity, this analysis focuses on diagonally-isotropic, time-invariant, and non-dispersive dielectrics within the waveguide. Under these stipulations, each set of fields and flux densities, (\textbf{E}, \textbf{D}), (\textbf{H}, \textbf{B}), can be related using the following constitutive relations
\begin{align}
	\textbf{D}=\epsilon\textbf{E}, \textbf{B}=\mu\textbf{H}
	\label{eq:cor}
\end{align}
with $\epsilon$ and $\mu$ as the permittivity and permeability of the dielectric respectively.

In this analysis, no fictitious magnetic conductors will be considered as they are not pertinent, thus $\textbf{M} = 0$. The free electric current density is treated as a linear superposition of Ohmic conduction $\textbf{J}_{Ohm}=\sigma\textbf{E}$ and and a source term $\textbf{J}_{src}$
\begin{align}
	\textbf{J} = \sigma\textbf{E}
	\label{eq:ohm}
\end{align}
where $\sigma$ is the diagonally-isotropic, time-invariant dielectric conductivity. In the described system, the inclusion of both a source current density and ohmic current density is not necessary as the wave is assumed to already be propagating in the waveguide from the TF/SF source, and Ohmic losses result in evanescent wave propagation along the waveguide's length. Despite this, these current density terms will be included in the governing set of equations for completeness. The full set of governing equations for waves propogating within the dielectric are as follows
\begin{align}
    \nabla\times\textbf{H} = \epsilon\frac{\partial\textbf{E}}{\partial t} + \sigma\textbf{E} + \textbf{J}_{src}
    \label{eq:ampere-final}
\end{align}
\begin{align}
    \nabla\times\textbf{E}=-\mu\frac{\partial\textbf{H}}{\partial t}.
    \label{eq:faraday-final}
\end{align}

Each of these 3D vector equations can be broken down into $\hat{x}$, $\hat{y}$, and $\hat{z}$ component equations; the $\hat{y}$ components of \textbf{E} and \textbf{H} in Eqs. (\ref{eq:ampere-final})-(\ref{eq:faraday-final}) are
\begin{align}
	\frac{\partial H_x}{\partial z} - \frac{\partial H_z}{\partial x} = \epsilon\frac{\partial E_y}{\partial t} + \sigma E_y + J_{y,src}
	\label{ampere-full-ey}
\end{align}
and
\begin{align}
	\frac{\partial E_x}{\partial z} - \frac{\partial E_z}{\partial x} =-\mu\frac{\partial H_y}{\partial t}
	\label{faraday-full-hy}
\end{align}
respectively.

These scalar equations are valid for all locations within the dielectric region excluding those inside of the PEC at which there is a Dirichlet boundary condition
\begin{align}
	E_x=E_y=0.
	\label{eq:dirichlet}
\end{align}
This Dirichlet boundary condition originates from the conservation of tangential electric fields at medium boundaries 
\begin{align}
	\hat{n}\times\textbf{E}_1=\hat{n}\times\textbf{E}_2.
	\label{eq:electricbc}
\end{align}
By nature of their infinite conductivity, electric fields cannot exist within in the PEC walls thus Eq. (\ref{eq:electricbc}) gives rise to Eq. (\ref{eq:dirichlet}).

\subsubsection{TF/SF 1-way Source}
\label{subsubsec:ftsf-mod}
One of the most popular formulations for injecting source fields into a simulation domain is via a TF/SF 1-way source \cite{taftlovefdtd}. The total-field, scattered-field formulation is built on the linearity of Maxwell's equations. Fields within the total-field region are a superposition of source fields and reflected fields where as fields in the scattered field only consist of those reflected off of materials within the simulation.

As shown in Fig. \ref{fig:model}, The TF/SF source is introduced in a plane with a normal vector $\hat{n}=-\hat{z}$. As outlined in \cite{taftlovefdtd}, fields may be introduced on such planes by fully specifying $E_x, E_y, H_x$ and $H_y$. For this analysis, the waveguide source fields will be restricted to $E_y, H_x$ and $H_z$ as in \cite{fieldsandwavescomm}. Thus for source fields originating on a $\hat{z}$ plane, only the former two fields need to be specified. 

To respect the Dirichlet boundary condition on PEC walls as defined in  Eqs. (\ref{eq:dirichlet}-\ref{eq:electricbc}), the spatial distribution of the steady state frequency solution must be satisfied as to ensure all numerical results are physical. These time independent solutions are as follows
\begin{align}
	E_y=E_0\sin{\frac{\pi x}{a}}
	\label{eq:eyfreq}
\end{align}
and
\begin{align}
	H_x = -\left(\eta\left[1-\left(\frac{\omega_c}{\omega}\right)^2\right]^{1/2}\right)^{-1}\left(E_0\sin{\frac{\pi x}{a}}\right)
	\label{eq:hxfreq}
\end{align}
where $E_0$ is the initial complex valued waveguide intensity, $a$ is the width of the waveguide as in \ref{fig:model}, $\eta$ is the intrinsic impedance of the waveguide dielectric, $\omega_c$ is the cutoff angular frequency of the waveguide, and $\omega$ is the angular frequency of the source field \cite{fieldsandwavescomm}.

Converting Eqs. (\ref{eq:eyfreq}-\ref{eq:hxfreq}) to the time-domain, the 1-way TF/SF source formulation becomes
\begin{align}
	E_y=E_{y,src}(t)\sin{\frac{\pi x}{a}} + E_{scat}
\end{align}
and
\begin{multline}
	H_x = -E_{y_src}(t)\left(\eta\left[1-\left(\frac{\omega_c}{\omega}\right)^2\right]^{1/2}\right)^{-1}\left(\sin{\frac{\pi x}{a}}\right) \\ + H_{scat}
\end{multline}
in the total field region with $E_{scat}$ as the scattered electric field, $H_{scat}$ as the scattered magnetic field, and $E_{y,src}$ as a time-varying $E_y$ source field. The time-varying source field is specific to the desired simulation outcomes. To obtain a response at a nearly monochromatic frequency, a tapered sine wave may be used 
\begin{align}
	E_{y,src} = E_0\left[1 - \exp{\frac{(t - t_d)}{\tau}}\right]\sin{\omega_0 t}
\end{align}
where $t_d$ is a delay time and $\tau$ is the temporal width of the ramping period. The tapered sine source gradually ramps-up to full field intensity reducing numerical artifacts from sudden jumps \cite{rothlecnotes}.

To obtain a wide-band simulation response, a modulated Gaussian pulse is ideal as it allows for the specification of frequency content via the temporal ramping period witdh $\tau$ and a carrier center angular frequency $\omega_0$ \cite{rothlecnotes}  
\begin{align}
	E_{y,src} = E_0\exp{\left(-\frac{1}{2}\left(\frac{t-t_d}{\tau}\right)^2\right)}\sin{\omega_0t}.
\end{align}

\subsubsection{Mur Absorbing Boundary Condition}
\label{subsub:murtheory}
Mur's absorbing boundary condition is a discretized form of the 1-way Engquist-Majda 1-way wave equation \cite{taftlovefdtd}, \cite{rothlecnotes}. A three-dimensional, elliptic wave equation describing the evolution of an arbitrary scalar field $U$ is given by
\begin{align}
	\frac{\partial^2 U}{\partial x^2}+\frac{\partial^2 U}{\partial y^2}+\frac{\partial^2 U}{\partial z^2}-\frac{1}{c}\frac{\partial^2 U}{\partial t^2}=0
	\label{eq:scalarelliptic}
\end{align}
as defined in \cite{taftlovefdtd}.

Via algebraic manipulation, and a second order accurate, Taylor series expansion of the general expression $\sqrt{1-s^2}\approx1-s^2/2$, the following continuous Engquist-Majda absorbing boundary condition at the $z=0$ plane can be derived from Eq. [\ref{eq:scalarelliptic}] as
\begin{align}
	\frac{\partial^2 U}{\partial z\partial t}-\frac{1}{c}\frac{\partial^2 U}{\partial t^2}+\frac{c}{2}\frac{\partial^2 U}{\partial x^2}+\frac{c}{2}\frac{\partial^2 U}{\partial y^2}=0
	\label{eq:enmaabc}
\end{align}
as in \cite{taftlovefdtd}.

To absorb the incident wave introduced in \ref{subsubsec:ftsf-mod}, Eq. [\ref{eq:enmaabc}] 
