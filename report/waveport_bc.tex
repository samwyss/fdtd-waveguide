\section{Waveport Boundary Condition}
\label{sec:wavebc}

The following is not necessarily written using the proper prose as mentioned in lecture. This is just a derivation, I will re-word this later.

Starting with (3.3.34) in Ch. 9 of our textbook
\begin{align}
    \textbf{E}(u,v,w) = \textbf{E}^{inc}(u,v,w) + \textbf{E}^{ref}(u,v,w).
\end{align}

Substituting in the $\textbf{e}_{10}$ definition in (9.3.35) yields the following
\begin{align}
    \textbf{E}(u,v,w) = \hat{v}E_0\sin{\frac{\pi u}{a}} e^{-j\beta w} + \hat{v}RE_0\sin{\frac{\pi u}{a}} e^{j\beta w}.
\end{align} 
where $R$ is the reflection coefficient and $a$ is the width of the waveguide.

This can be cleanly written in vector notation as
\begin{multline}
    \textbf{E}(u,v,w) =\begin{bmatrix}
        0 \\
        E_0\sin{\frac{\pi u}{a}}(e^{-j\beta w}+Re^{j\beta w})\\
        0
    \end{bmatrix}  \\ =\begin{bmatrix}
        0 \\
        c(u)(e^{-j\beta w}+Re^{j\beta w})\\
        0
    \end{bmatrix}.
\end{multline}

With 
\begin{align}
    \textbf{E}^{inc}(u,v,w) =\begin{bmatrix}
        0 \\
        c(u)e^{-j\beta w}\\
        0
    \end{bmatrix}.
\end{align}

The waveport source boundary condition as found in (9.3.37) is 
\begin{align}
    \hat{n}\times(\nabla\times\textbf{E})+j\beta\hat{n}\times(\hat{n}\times\textbf{E})=-2j\beta\textbf{E}^{inc}.
    \label{wavbc}
\end{align}

Substituting the above field definitions into \ref{wavbc} yields the following
\begin{multline}
    \hat{n} \times \left(\nabla \times \begin{bmatrix}
        0 \\
        c(u)(e^{-j\beta w}+Re^{j\beta w})\\
        0
    \end{bmatrix}\right) \\ +j\beta\hat{n}\times\left(\hat{n} \times \begin{bmatrix}
        0 \\
        c(u)(e^{-j\beta w}+Re^{j\beta w})\\
        0
    \end{bmatrix}\right) \\ =-2j\beta\begin{bmatrix}
        0 \\
        c(u)e^{-j\beta w}\\
        0
    \end{bmatrix}.
    \label{wavbcvec}
\end{multline}

Assuming $\hat{u},\hat{v},\hat{w}$ form an orthogonal, right-handed set of basis vectors with $\hat{n}=\hat{w}$ as it is normal to the waveport surface in the direction of propagation \ref{wavbcvec} becomes
\begin{multline}
    \begin{bmatrix}
        0 \\ 0 \\ 1
    \end{bmatrix} \times \left(
        \begin{bmatrix}
            \partial_u \\ 
            \partial_v \\
            \partial_w \\ 
        \end{bmatrix} \times 
    \begin{bmatrix}
        0 \\
        c(u)(e^{-j\beta w}+Re^{j\beta w})\\
        0
    \end{bmatrix}\right) \\ +j\beta 
    \begin{bmatrix}
        0 \\ 0 \\ 1
    \end{bmatrix} \times \left(
        \begin{bmatrix}
            0 \\ 0 \\ 1
        \end{bmatrix} \times 
    \begin{bmatrix}
        0 \\
        c(u)(e^{-j\beta w}+Re^{j\beta w})\\
        0
    \end{bmatrix}\right) \\ =-2j\beta
    \begin{bmatrix}
        0 \\
        c(u)e^{-j\beta w}\\
        0
    \end{bmatrix}.
\end{multline}

Evaluating the first set of cross products results in 
\begin{multline}
    \begin{bmatrix}
        0 \\ 0 \\ 1
    \end{bmatrix} \times
    \begin{bmatrix}
        -\partial_w(c(u)(e^{-j\beta w}+Re^{j\beta w})) \\
        0 \\
        \partial_u(c(u)(e^{-j\beta w}+Re^{j\beta w}))
    \end{bmatrix} \\ +j\beta 
    \begin{bmatrix}
        0 \\ 0 \\ 1
    \end{bmatrix} \times
    \begin{bmatrix}
        -c(u)(e^{-j\beta w}+Re^{j\beta w}) \\
        0\\
        0
    \end{bmatrix} \\ =-2j\beta
    \begin{bmatrix}
        0 \\
        c(u)e^{-j\beta w}\\
        0
    \end{bmatrix}.
\end{multline}

Evaluating the second set of cross products equals
\begin{multline}
    \begin{bmatrix}
        0 \\
        -\partial_w(c(u)(e^{-j\beta w}+Re^{j\beta w})) \\
        0 \\ 
    \end{bmatrix} \\ +j\beta 
    \begin{bmatrix}
        0  \\
        -c(u)(e^{-j\beta w}+Re^{j\beta w})\\
        0
    \end{bmatrix} \\ =-2j\beta
    \begin{bmatrix}
        0 \\
        c(u)e^{-j\beta w}\\
        0
    \end{bmatrix}.
\end{multline}

Resulting in the following expression for the $\hat{v}$ direction
\begin{multline}
    -\partial_w(c(u)(e^{-j\beta w}+Re^{j\beta w})) \\ - j\beta c(u)(e^{-j\beta w}+Re^{j\beta w}) \\ =-2j\beta c(u)e^{-j\beta w}.
\end{multline}

Which is equal to
\begin{align}
    \partial_w E_v+j\beta E_v =2j\beta E_v^{inc}.
\end{align}

Substituting in the definition of beta yields

\begin{align}
    \partial_w E_v+j\sqrt{k^2-\frac{\pi^2}{a^2}} E_v =2j\sqrt{k^2-\frac{\pi^2}{a^2}} E_v^{inc}.
\end{align}

Pulling out $k=\frac{\omega\sqrt{\mu_r\epsilon_r}}{c_0}$ from square roots yields

\begin{multline}
    \partial_w E_v+j\omega\frac{\sqrt{\mu_r\epsilon_r}}{c_0}\sqrt{1-\frac{\pi^2}{k^2a^2}} E_v \\ =2j\omega\frac{\sqrt{\mu_r\epsilon_r}}{c_0}\sqrt{1-\frac{\pi^2}{k^2a^2}} E_v^{inc}.
\end{multline}

Using the second order approximation of $\partial_w E_v\approx-j\omega\frac{\sqrt{\mu_r\epsilon_r}}{c_0}\sqrt{1-\frac{\pi^2}{k^2a^2}}E_v$ the above becomes
\begin{multline}
    -j\omega\frac{\sqrt{\mu_r\epsilon_r}}{c_0}\sqrt{1-\frac{\pi^2}{k^2a^2}}E_v+j\omega\frac{\sqrt{\mu_r\epsilon_r}}{c_0}\sqrt{1-\frac{\pi^2}{k^2a^2}} E_v \\ =2j\omega\frac{\sqrt{\mu_r\epsilon_r}}{c_0}\sqrt{1-\frac{\pi^2}{k^2a^2}} E_v^{inc}.
\end{multline}

Canceling out like terms gives
\begin{align}
    E_v = E_v^{inc}.
\end{align}

Which feels incorrect but is true for $R=0$.