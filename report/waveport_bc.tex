\section{Waveport Boundary Condition}
\label{sec:wavebc}

The following is not necessarily written using the proper prose as mentioned in lecture. This is just a derivation, I will re-word this later.

Starting with (3.3.34) in Ch. 9 of our textbook
\begin{align}
    \textbf{E}(u,v,w) = \textbf{E}^{inc}(u,v,w) + \textbf{E}^{ref}(u,v,w).
\end{align}

Substituting in the $\textbf{e}_{10}$ definition in (9.3.35) yields the following
\begin{align}
    \textbf{E}(u,v,w) = \hat{v}E_0\sin{\frac{\pi u}{a}} + \hat{v}RE_0\sin{\frac{\pi u}{a}}.
\end{align} 
where $R$ is the reflection coefficient and $a$ is the width of the waveguide.

This can be cleanly written in vector notation as
\begin{align}
    \textbf{E}(u,v,w) =\begin{bmatrix}
        0 \\
        E_0\sin{\frac{\pi u}{a}}(1+R)\\
        0
    \end{bmatrix} =\begin{bmatrix}
        0 \\
        c(u)(1+R)\\
        0
    \end{bmatrix}.
\end{align}

With 
\begin{align}
    \textbf{E}^{inc}(u,v,w) =\begin{bmatrix}
        0 \\
        c(u)\\
        0
    \end{bmatrix}.
\end{align}

The waveport source boundary condition as found in (9.3.37) is 
\begin{align}
    \hat{n}\times(\nabla\times\textbf{E})+j\beta\hat{n}\times(\hat{n}\times\textbf{E})=-2j\beta\textbf{E}^{inc}.
    \label{wavbc}
\end{align}

Substituting the above field definitions into \ref*{wavbc} yields the following
\begin{multline}
    \hat{n} \times \left(\nabla \times \begin{bmatrix}
        0 \\
        c(u)(1+R)\\
        0
    \end{bmatrix}\right) \\ +j\beta\hat{n}\times\left(\hat{n} \times \begin{bmatrix}
        0 \\
        c(u)(1+R)\\
        0
    \end{bmatrix}\right) \\ =-2j\beta\begin{bmatrix}
        0 \\
        c(u)\\
        0
    \end{bmatrix}.
    \label{wavbcvec}
\end{multline}

Assuming $\hat{u},\hat{v},\hat{w}$ form an orthogonal, right-handed set of basis vectors with $\hat{n}=\hat{w}$ as it is normal to the waveport surface in the direction of propagation \ref*{wavbcvec} becomes
\begin{multline}
    \begin{bmatrix}
        0 \\ 0 \\ 1
    \end{bmatrix} \times \left(
        \begin{bmatrix}
            \partial_x \\ 
            \partial_y \\
            \partial_z \\ 
        \end{bmatrix} \times 
    \begin{bmatrix}
        0 \\
        c(u)(1+R)\\
        0
    \end{bmatrix}\right) \\ +j\beta 
    \begin{bmatrix}
        0 \\ 0 \\ 1
    \end{bmatrix} \times \left(
        \begin{bmatrix}
            0 \\ 0 \\ 1
        \end{bmatrix} \times 
    \begin{bmatrix}
        0 \\
        c(u)(1+R)\\
        0
    \end{bmatrix}\right) \\ =-2j\beta
    \begin{bmatrix}
        0 \\
        c(u)\\
        0
    \end{bmatrix}.
\end{multline}.

Evaluating the first set of cross products results in 
\begin{multline}
    \begin{bmatrix}
        0 \\ 0 \\ 1
    \end{bmatrix} \times
    \begin{bmatrix}
        0 \\
        0 \\
        \partial_u(c(u)(1+R))
    \end{bmatrix} \\ +j\beta 
    \begin{bmatrix}
        0 \\ 0 \\ 1
    \end{bmatrix} \times
    \begin{bmatrix}
        -c(u)(1+R) \\
        0\\
        0
    \end{bmatrix} \\ =-2j\beta
    \begin{bmatrix}
        0 \\
        c(u)\\
        0
    \end{bmatrix}.
\end{multline}

this creates a problem as the spatial derivative drops