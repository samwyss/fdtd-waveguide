\section{Numerical Results}
\label{sec:numres} 
All update equations as defined in Section \ref{subsec:timestepeqs} were implemented in Rust. This language was choosen for its C++ like performance while enforcing compile-time memory safety which makes writing fast and safe CEM codes relatively easy. An overview of this implementation can be found in \ref{subsec:code}.

\subsection{Verification and Validation}
\label{subsec:vv}
To verify and validate model results, the case of an infinitely long waveguide with fields propagating in the $\mathrm{TE_{10}}$ mode. This case is chosen as simulated results can easily be compared to analytic results.

All verification and validation analyses are performed with $0.1$m stretch of a WR-90, X-band waveguide with a cross section of $a=0.02286$m and $b=0.01016$m \cite{everythingrf}. Said configuration has an analytic $\mathrm{TE_{10}}$ cutoff frequency of as calculated by
\begin{align}
    f_c=\frac{c}{2\pi\sqrt{\epsilon_r\mu_r}}\sqrt{\bigg[\bigg]^2}
    \label{eq:analytic-cutoff}
\end{align}
from \cite{pozar2011microwave}.

\subsection{Analysis of Unloaded Q with Varying Dielectric Loss}
\label{subsec:dielectric-loss}

\subsection{Analysis of Resonance Frequency with Varying Resonator Length}
\label{subsec:resonance-length}
